\documentclass[11pt]{article}
\usepackage{graphicx} % more modern
%\usepackage{times}
\usepackage{helvet}
\usepackage{courier}
\usepackage{epsf}
\usepackage{amsmath,amssymb,amsfonts,verbatim}
\usepackage{subfigure}
\usepackage{amsfonts}
\usepackage{amsmath}
\usepackage{latexsym}
\usepackage{algpseudocode}
\usepackage{algorithm}
\usepackage{enumerate}
%\usepackage{algorithmic}
\usepackage{multirow}
\usepackage{xcolor}
\usepackage[left=2cm,right=2cm,top=2cm,bottom=2cm]{geometry}
\usepackage[utf8]{inputenc}
\usepackage{amsthm}

\def\A{{\bf A}}
\def\a{{\bf a}}
\def\B{{\bf B}}
\def\b{{\bf b}}
\def\C{{\bf C}}
\def\c{{\bf c}}
\def\D{{\bf D}}
\def\d{{\bf d}}
\def\E{{\bf E}}
\def\e{{\bf e}}
\def\F{{\bf F}}
\def\f{{\bf f}}
\def\G{{\bf G}}
\def\g{{\bf g}}
\def\k{{\bf k}}
\def\K{{\bf K}}
\def\H{{\bf H}}
\def\I{{\bf I}}
\def\L{{\bf L}}
\def\M{{\bf M}}
\def\m{{\bf m}}
\def\n{{\bf n}}
\def\N{{\bf N}}
\def\BP{{\bf P}}
\def\R{{\bf R}}
\def\BS{{\bf S}}
\def\s{{\bf s}}
\def\t{{\bf t}}
\def\T{{\bf T}}
\def\U{{\bf U}}
\def\u{{\bf u}}
\def\V{{\bf V}}
\def\v{{\bf v}}
\def\W{{\bf W}}
\def\w{{\bf w}}
\def\X{{\bf X}}
\def\Y{{\bf Y}}
\def\Q{{\bf Q}}
\def\x{{\bf x}}
\def\y{{\bf y}}
\def\Z{{\bf Z}}
\def\z{{\bf z}}
\def\0{{\bf 0}}
\def\1{{\bf 1}}


\def\hx{\hat{\bf x}}
\def\tx{\tilde{\bf x}}
\def\ty{\tilde{\bf y}}
\def\tz{\tilde{\bf z}}
\def\hd{\hat{d}}
\def\HD{\hat{\bf D}}

\def\MA{{\mathcal A}}
\def\MF{{\mathcal F}}
\def\MR{{\mathcal R}}
\def\MG{{\mathcal G}}
\def\MI{{\mathcal I}}
\def\MN{{\mathcal N}}
\def\MO{{\mathcal O}}
\def\MT{{\mathcal T}}
\def\MX{{\mathcal X}}
\def\SW{{\mathcal {SW}}}
\def\MW{{\mathcal W}}
\def\MY{{\mathcal Y}}
\def\BR{{\mathbb R}}
\def\BP{{\mathbb P}}

\def\bet{\mbox{\boldmath$\beta$\unboldmath}}
\def\epsi{\mbox{\boldmath$\epsilon$}}

\def\etal{{\em et al.\/}\,}
\def\tr{\mathrm{tr}}
\def\rk{\mathrm{rk}}
\def\diag{\mathrm{diag}}
\def\dg{\mathrm{dg}}
\def\argmax{\mathop{\rm argmax}}
\def\argmin{\mathop{\rm argmin}}
\def\vecd{\mathrm{vec}}

\def\ph{\mbox{\boldmath$\phi$\unboldmath}}
\def\vp{\mbox{\boldmath$\varphi$\unboldmath}}
\def\pii{\mbox{\boldmath$\pi$\unboldmath}}
\def\Ph{\mbox{\boldmath$\Phi$\unboldmath}}
\def\pss{\mbox{\boldmath$\psi$\unboldmath}}
\def\Ps{\mbox{\boldmath$\Psi$\unboldmath}}
\def\muu{\mbox{\boldmath$\mu$\unboldmath}}
\def\Si{\mbox{\boldmath$\Sigma$\unboldmath}}
\def\lam{\mbox{\boldmath$\lambda$\unboldmath}}
\def\Lam{\mbox{\boldmath$\Lambda$\unboldmath}}
\def\Gam{\mbox{\boldmath$\Gamma$\unboldmath}}
\def\Oma{\mbox{\boldmath$\Omega$\unboldmath}}
\def\De{\mbox{\boldmath$\Delta$\unboldmath}}
\def\de{\mbox{\boldmath$\delta$\unboldmath}}
\def\Tha{\mbox{\boldmath$\Theta$\unboldmath}}
\def\tha{\mbox{\boldmath$\theta$\unboldmath}}

\newtheorem{theorem}{Theorem}[section]
\newtheorem{lemma}{Lemma}[section]
\newtheorem{definition}{Definition}[section]
\newtheorem{proposition}{Proposition}[section]
\newtheorem{corollary}{Corollary}[section]
\newtheorem{example}{Example}[section]

\def\probin{\mbox{\rotatebox[origin=c]{90}{$\vDash$}}}

\def\calA{{\cal A}}



%this is a comment

%use this as a template only... you may not need the subsections,
%or lists however they are placed in the document to show you how
%do it if needed.


%THINGS TO REMEMBER
%to compile a latex document - latex filename.tex
%to view the document        - xdvi filename.dvi
%to create a ps document     - dvips filename.dvi
%to create a pdf document    - dvipdf filename.dvi
%{\bf TEXT}                  - bold font TEXT
%{\it TEXT}                  - italic TEXT
%$ ... $                     - places ... in math mode on same line
%$$ ... $$                   - places ... in math mode on new line
%more info at www.cs.wm.edu/~mliskov/cs423_fall04/tex.html


\setlength{\oddsidemargin}{.25in}
\setlength{\evensidemargin}{.25in}
\setlength{\textwidth}{6in}
\setlength{\topmargin}{-0.4in}
\setlength{\textheight}{8.5in}


%%%%%%%%%%%%%%%%%%%%%%%%%%%%%%%%%%%%%%%%%%%%%%%%%%%%%%%%%%%%%%%%%%%%%%%%%%%%%%%%%%%
\newcommand{\notes}[5]{
	\renewcommand{\thepage}{#1 - \arabic{page}}
	\noindent
	\begin{center}
	\framebox{
		\vbox{
		\hbox to 5.78in { { \bf Numerical Linear Algebra}
		\hfill #2}
		\vspace{4mm}
		\hbox to 5.78in { {\Large \hfill #5 \hfill} }
		\vspace{2mm}
		\hbox to 5.78in { {\it #3 \hfill #4} }
		}
	}
	\end{center}
	\vspace*{4mm}
}

\newcommand{\ho}[5]{\notes{#1}{}{Professor: Zhihua Zhang}{}{Lecture Notes #1:Applications}}
%%%%%%%%%%%%%%%%%%%%%%%%%%%%%%%%%%%%%%%%%%%%%%%%%%%%%%%%%%%%%%%%%%%%%%%%%%%%%%%%%%

%begins a LaTeX document
\setcounter{section}{0}
\setcounter{subsection}{0}
\setcounter{theorem}{0}
\setcounter{definition}{0}
\setcounter{example}{0}
\begin{document}
\ho{6}{2014.03.08}{Moses Liskov}{Name}{Lecture title}


\begin{theorem}
 If $A$ and $B$ are $n \times n$ matrices, then 
 \begin{itemize}
  \item $\sigma(A+B) \prec_w \sigma(A) + \sigma(B)$ 
  \item $\sum_{i=1}^k \sigma_i(A) - \sum_{i=1}^k \sigma_{n-i+1}(B)$, $k=1,...,n$
  \item If $A = [A_1, A_2]$, $A_1\in\BR^{n\times p}$, $A_2 \in \BR^{n\times(n-p)}$, and $B = [A_1, 0]$, then $\sigma(A) \succ_w \sigma(B)$.
 \end{itemize}
\end{theorem}

\begin{definition}
 A real-valued function $\|\cdot\|$ on $\BR^{m\times n}$ is called a unitarily invariant norm if the following conditions are satisfied 
 \begin{enumerate}
  \item $\|X\| > 0$, $X \neq 0$
  \item $\|\alpha X\| = |\alpha| \|X\|$
  \item $\|X + Y\| \leq \|X\| + \| Y \|$
  \item $\|XV\| = \|UX\| = \|X\|$. Here $U$ and $V$ are any unitary matrix.
 \end{enumerate}
\end{definition}

Conduct SVD decomposition $X = U\Sigma V^T$, then we have
\begin{itemize}
 \item $\|X\| = \|U\Sigma V^T\| = \|\Sigma\| = \|\sigma(X)\|$
 \item $\|\sigma(X)\|_2 = \|X\|_F$
 \item $\|\sigma(X)\|_\infty = |||X|||_2$.
\end{itemize}

\begin{theorem}
 Let $A$ and $B$ be $m\times n$ matrices, then for all $n\times n$ matrices $X$,
 \[ \|A(A^\dagger B) - B\| \leq \| AX - B \| \]
 for every unitarily invariant norm $\|\cdot\|$.
\end{theorem}
\begin{proof}
 Here is an easy proof from symmetric gauge function.
 \begin{equation*}
  \begin{split}
	A^\dagger B &= \underset{X\in\BR^{n\times n}}{\arg\min} \|AX - B\| \\
				&\Leftrightarrow \phi(\sigma(AA^\dagger B - B )) \leq \phi(\sigma(AX - B)) \\
				&\Leftrightarrow \sigma(AA^\dagger B - B ) \prec_w \sigma(AX - B) 
  \end{split}
 \end{equation*}
	Define $L = AX - B$, $P = AX - AA^\dagger B$, $Q = AA^\dagger B - B$. It is easy to verify $L = P + Q$ and $P^T Q = 0$.
	Thus we have 
	\[ L^T L = P^TP + Q^T Q, \]
	which indicates that $\lambda_i(L^T L) \leq \lambda_i(Q^T Q) \Leftrightarrow \sigma_i(L) \leq \sigma_i(Q)$.
\end{proof}



\begin{theorem}[Eckart and Young 1936, Minksy 1960]
 Let $A$ be an $m\times n$ matrix of rank $r$. Let $A = U_A \Sigma_S V_A^T$ be the full SVD and let $k \leq r$. 
 Define $A_k = U_{A,k} \Sigma_{A,k} V_{A,k}^T$.
 Then $\|A - A_k\| \leq \|A - X\|$ for all $m\times n$ matrices $X$ of rank $k$ and every unitarily invariant norm.
\end{theorem}

\section{Subgradient of Matrix}

Let us consider an optimization problem,
\[ \min f(X) \triangleq \|A - X\|_F^2 + \lambda \| X \|_*. \]

First, let us review the gradient of scalar value and vector,
\begin{itemize}
\item $f : \BR \to \BR$, $\partial f(x) = \{ z \in \BR : f(y) \geq f(x) + (y-x)z \quad \forall y\in\BR \}$ 
\item $f : \BR^n \to \BR$, $\partial f(x) = \{z\in\BR^n: f(y) \geq f(x) + (y-x)^Tz\quad\forall y\in\BR^n\}$
\end{itemize}

\begin{definition}
 Let $\|\cdot\|$ be a norm on $\BR^{m \times n}$ real-value matrix, and $A \in \BR^{m \times n}$.
 Then subgradient of $\|A\|$ is defined by 
 \[ \partial \|A\| = \{G\in\BR^{m \times n} : \|B\| \geq \|A\| + tr((B-A)^TG) \quad \forall B\in\BR^{m \times n} \} \].
\end{definition}

\begin{lemma}
 $G \in \partial \|A\|$ is equivalent to that 
\begin{enumerate}
	\item $\|A\| = tr(G^TA)$
	\item $\|G\|_D \leq 1$
\end{enumerate}
\end{lemma}
\begin{proof}(Not complete)
  If $G\in\partial \|A\|$, then $\forall B\in \BR^{m \times n}$, $\|B\| \geq \|A\| + tr((B-A)^T G)$.
  Let $B = 2A$, then we have 
  \[ 2\|A\| \geq \|A\| + tr(A^T G) \implies \|A\| \geq tr(A^T G). \]
  Let $B = \frac{1}{2}A$, then we have $ \|A\| \leq tr(A^T G) $ similarily. Then we get $ \|A\| = tr(G^TA) $.
  This implies that 
  \[ tr(B^T G) \leq \|B\| \quad \forall B. \]
  Then we have
  \[ \|G\|_D = \max_{\|Z\| = 1} tr(G^TZ) \leq \|Z\| = 1.\]
  
  Next we prove another direction.
  $\|\cdot\|$ is a unitarily invariant norm, then there exists a symmetric gauge function $\phi(\sigma(A)) = \|A\|$.\newline
  $ Z \in \partial \phi(X)$ is equivalent to 
  \begin{enumerate}
   \item $\phi(x) = x^T z$
   \item $\phi^*(z) \leq 1$, where $\phi*(z) = \max_{\phi(y) = 1} z^Ty$.
  \end{enumerate}
\end{proof}

\begin{theorem}
 Let $\|\cdot\|$ be unitarily invariant norm on $\BR^{m \times n}$, $D = diag(\vec{d})$ and $\vec{d} = dg(D)$.
 Then $\partial \|A\| = conv\{ UDV^T : A = U\Sigma V^T, \vec{d} \in \partial \phi(\vec{\sigma}) \}$. 
\end{theorem}

% \begin{proof}
% 	If $G \in \partial\|A\|$, then $G = \sum \lambda_i U_i D_i V_i^T$. It is equivalent to prove 
%   $tr(G^T A) = \|A\|$ and $\|G\|_D \leq 1$.
%   
%   \begin{equation*}
%   \begin{split}
%    tr(G^TA) 	&= tr(A^T \sum \lambda_iU_iD_i V_i^T)  \\
% 				&= tr(A^T \sum \lambda_i U P_i D_i P_i^T V^T)  \\
% 				&= tr(V\Sigma U^T \sum \lambda_i U P_i D_i P_i^T V^T)  \\
% 				&= tr( \Sigma \sum \lambda_i P_i D_i P_i^T )\\
% 				&= \sum tr(\Sigma \lambda_i P_i D_i P_i^T) \\
% 				&= \sum \lambda_i tr(\Sigma D_i) \\
% 				&= \sum \lambda_i \vec{\sigma}^T \vec{d_i} \\
% 				&= \sum \lambda_i \phi(\sigma)
%    \end{split}
%   \end{equation*}
% 
%   
%   
%   
% 
% \end{proof}


\end{document}